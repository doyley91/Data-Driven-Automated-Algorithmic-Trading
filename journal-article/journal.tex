%&platex --translate-file=cp1250pl
\documentclass[times]{jtitauth}

\begin{document}

\title{Data Driven Automated Algorithmic Trading}

\author{Gabriel Gauci Maistre}

\markboth{Gabriel Gauci Maistre}
{Data Driven Automated Algorithmic Trading}

\maketitle

\begin{abstract}
Various existing stock market price forecasting methods were analysed in this report. Three methods were applied towards the problem making use of Technical Analysis, these were Time Series Analysis, Machine Learning, and Bayesian Statistics. Through the results of this report, it was found that the Efficient Market Hypothesis remains true, that past data does not contain enough useful information to forecast future prices and gain an advantage over the market. However, the results proved that Technical Analysis and Machine Learning could still be used to guide an investors decision. It was also found that the Random Walk Hypothesis was not necessarily true, as some stocks showed signs of auto and partial correlation. A common application of technical analysis was demonstrated and shown to produce limited useful information in beating the market. Based on the findings, a number of automated trading algorithms were developed using machine learning and backtested to determine their effectiveness.
\end {abstract}

\begin{keywords}
machine learning, time series analysis,
probabilistic, bayesian, statistics,
inference\
\end{keywords}

\section{Introduction}

The stock market retains its status as a prime location for investors to invest in the market and earn a profit, however this is not always easy due to the constantly thriving and changing nature which follows the stock market. Investors are constantly presented with numerous profit potential opportunities, however without intensive planning and analysis, these opportunities could easily turn into losses. This means that it is crucial for every investor to carry out stock market ananlysis prior to any investment by monitoring past price movements in order to forecast future trends. Even though past data is not a clear indication of future movement, it is still proven to provide some useful insite.

\section{Related work}
Darrat et al. set out to investigate with the use of new daily data, whether prices in the two Chinese stock exchanges (Shanghai and Shenzhen) follow a random walk process as required by market efficiency. Two different approaches were applied, the standard variance-ratio test, and a model-comparison test that compares the ex post forecasts from a naive model with those obtained from several alternative models such as ARIMA, GARCH, and ANNs. To evaluate ex post forecasts, Darrat et al. made use of several procedures including root-mean-square error (RMSE), mean absolute error (MAE), uncertainty coefficient, and encompassing tests. It was concluded that the model-comparison approach yielded results which were quite strongly rejected the RWH in both Chinese stock markets when compared with the variance-ratio test. Darret et al. recommended the use of ANNs, as their results showed strong support for the model
as a potentially useful factor for forecasting stock prices in emerging markets.

Support vector machines (SVM), are a class of machine learning algorithms that have become incredibly popular in the past few years. SVMs are very similar to classifiers in the sense that they also classify data by drawing a line, called a decision boundary, to separate them. However, SVMs go a step further by calculating a vector from the data point with the smallest margin to the decision boundary. This is called a support vector. A vast majority of academics tend to predict the price of stocks in financial markets, however most models used are flawed and only focus on the accurate forecasting of the levels of the underlying stock index. There is a lack of studies examining the predictabil- ity of the direction of stock index movement. Given the notion that a prediction with little forecast error does not necessarily translate into capital gain, the authors of this research attempt to predict the direction of the S&P CNX NIFTY Market Index of the National Stock Exchange, one of the fastest growing financial exchanges in developing Asian countries. Machine learning models such as random forest and SVMs, differ widely from other models, and are making strides in predicting the financial markets. Kumar et al. tested classification models to predict the direction of the markets, by applying models such as linear discriminant analysis, logistic regression, ANNs, random forest, and SVM. Their evidence shows that SVMs outperform the other classification methods in terms of predicting the direction of the stock market movement, and that the random forest model outperforms other models such as ANNs, discriminant analysis, and logistic regression.

\section{Conclusion}
Three financial forecasting methods were presented in this report, two of which showed little to no potential of ever producing any statistically significant result when the correct methodology was applied. The third method, machine learning, showed some potential in the tests carried out, which is why this method was built into an automated algo- rithmic strategy to trade with. The algorithm proved to be successful in forecasting future prices, using both classification and regression methods. However, the backtest- ing proved this method to fail in forecasting price falls. Once this factor was removed from the equation, the algorithms were very successful and reported a profit by the end of the test. This is however not always ideal as stocks which could fall in price could be catastrophic to the strategy. A stop loss would be ideal in insuring that no positions are held in downward falling stocks. It was also evident that regression methods were more successful in forecasting future price movements when compared to classification methods.

If there is anything that this report shows, is that profitable stock market prediction is an extremely tough problem. Even though the strategies reported a profit by the end of the backtest, they still did not beat the market. Whether it is at all possible to use such methods to outperform the market’s returns, ultimately remains an open question. These findings support the Efficient Market Hypothesis, proving that casual investors are better off investing in passive buy and hold strategies consisting of index funds and ETFs. However, there was some evidence found showing that the Random Walk Hypothesis does not hold true for all cases, as some stocks did show signs of repeating trends.

\section*{Acknowledgements}

I would like to express my special thanks of gratitude to my supervisor, Alan Gatt, for the patient guidance, encouragement, and advice he has provided throughout my time as his student. I would also like to thank Luke Vella Critien, for guiding me towards the right parth in the early stages of my research and for recommending Alan Gatt as my tutor. My gratitude is also extended to Emma Galea and Miguel Attard for their valuable input while carrying out my research. Completing this work would have been all the more difficult were it not for the support and friendship provided by the other members of the Malta College of Arts, Sciences, and Technology, and the institute of Information and Technology. I am indebted to them for their help. Finally, I would like to thank my family who have supported me all throughout the final year of my bachelor’s.

\begin{thebibliography}{99}
\bibitem{1}K. R. Wilson and V. V. Yakovlev, ,,Ultrafast rainbow: tunable ultrashort
 pulses from a solid-state kilohertz system'', {\it J. Opt. Soc. Am. B}, vol. 14, pp. 444--448, 1997.
\bibitem{2}J. Comly and E. Garmire, ,,Second harmonic generation from short
pulses'', {\it Appl. Phys. Lett.}, vol. 12, no. 7-9, 1968.
\bibitem{3}O. E. Martinez, ,,Achromatic phase matching for second harmonic ge\-neration
of femtosecond pulses'', {\it IEEE J. Quant. Electron.}, vol.
QE-25, pp. 2464--2468, 1989.
\bibitem{4}G. Szabo and Z. Bor, ,,Broadband frequency doubler for
femtosecond pulses'', {\it Appl. Phys. B}, vol. 50, pp. 51--54,
1990.
\bibitem{5}J.-Y. Zhang, J. Y. Huang, H. Wang, K. S. Wong, and G.K. Wong, ,,Second-harmonic
generation from regeneratively amplified femtosecond laser pulses
in BBO and LBO crystals", {\it J. Opt. Soc. Am. B}, vol. 15,
pp.~200--209, 1998.
\bibitem{6}K. Hayata and M. Koshiba,
,,Group-velocity-matched second-harmonic generation: an efficient
scheme for femtosecond ultraviolet pulse gene\-ration in
periodically domain-inverted $\beta$-BaB2O4", {\it Appl. Phys.
Lett.}, vol.~62, pp.~2188--2190, 1993.
\bibitem{7}G. Y. Wang and E. M. Garmire, ,,High-efficiency generation of ultrashort
second-harmonic pulses based on the Cherenkov geometry", {\it Opt.
Lett.}, vol. 19, pp. 254--256, 1994.
\bibitem{8}C. Radzewicz, Y. B. Band, G. W. Pearson, and J. S. Krasinski, ,,Short pulse
nonlinear frequency conversion without group-velocity-mismatch
broadening", {\it Opt. Commun.}, vol. 117, pp. 295--303, 1995.
\bibitem{9}P. Di Trapani, A. Andreoni, G. P. Banfi, C. Solcia, R. Danielius, A.~Piskarskas,
P. Foggi, M. Monguzzi, and C. Sozzi, ,,Group-velocity
self-matching of femtosecond pulses in noncollinear parametric
generation", {\it Phys. Rev. A}, vol. 51, pp. 3164--3168, 1995.
\bibitem{10}V. Krylov, A. Kalintsev, A. Rebane, D. Erni, and U. P.
Wild, ,,Noncollinear parametric generation in LiIO3 and
$\beta$-barium borate by frequency-doubled femtosecond Ti:sapphire
laser pulses", {\it Opt. Lett.}, vol.~20, pp.~151--153, 1995.
\bibitem{11}P. Di Trapani, A. Andreoni, C. Solcia, P. Foggi, R.
Danielius, A. Dubietis, and A. Piskarskas, ,,Matching of group
velocities in three-wave parametric interaction with fs pulses and
application to travelling-wave generators", {\it J. Opt. Soc. Am.
B}, vol. 12, pp. 2237--2244, 1995.
\bibitem{12}P. Di Trapani, A. Andreoni, P. Foggi, C. Solcia, R.
Danielius, and A.~Piskarskas, ,,Efficient conversion of
femtosecond blue pulses by travelling-wave parametric generation
in non-collinear phase matching", {\it Opt. Commun.}, vol. 119,
pp. 327--332, 1995.
\bibitem{13}T. R. Zhang, H. R. Choo, and M. C. Downer, ,,Phase and group
velocity matching for second harmonic generation of femtosecond
pulses", {\it Appl. Opt.}, vol. 29, pp. 3927--3933, 1990.
\bibitem{14}A. Andreoni and M. Bondani, ,,Group-velocity control in the
\mbox{mixing} of three non-collinear phase-matched waves", {\it
Appl. Opt.}, vol.~37, \mbox{pp.~2414--2423,} 1998.
\bibitem{15}P. Di Trapani, A. Andreoni, C. Solcia, G. P. Banfi, R.
Danielius, A.~Piskarskas, and P. Foggi, ,,Powerful sub-100-fs
pulses broadly tunable in the visible from a blue-pumped
parametric ganerator and amplifier", {\it J. Opt. Soc. Am. B},
vol. 14, pp. 1245--1248, 1997.
\bibitem{16}R. Danielius, A. Piskarskas, P. Di Trapani, A. Andreoni,
C. Solcia, and P. Foggi, ,,Matching of group velocities by spatial
walk-off in collinear three-wave interaction with tilted pulses",
{\it Opt. Lett.}, vol. 21, \mbox{pp.~973--975,} 1996.
\bibitem{17}R. Danielius, A. Piskarskas, A. Stabinis, G. P. Banfi, P.
Di Trapani, and R. Righini, ,,Travelling-wave parametric
generation of widely tunable, highly coherent femtosecond light
pulses", {\it J. Opt. Soc. Am. B}, vol. 10, pp. 2222--2232, 1993.
\bibitem{18}S. A. Akhmanov, A. S. Chirkin, K. N. Drabovich, A. I.
Kovrigin, R.~V.~Khokhlov, and A. P. Sukhorukov, ,,Nonstationary
nonlinear optical effects and ultrashort light pulse formation",
{\it IEEE J. Quant. Ele\-ctron.}, vol. QE-4, pp. 598--605, 1968.
\bibitem{19}R. Danielius, A. Piskarskas, P. Di Trapani, A. Andreoni,
C. Solcia, and P. Foggi, ,,A collinearly phase-matched parametric
generator/amplifier of visible femtosecond pulses", {\it IEEE J.
Quant. Electron.}, vol. 34, \mbox{pp.~459--464,} 1998.
\bibitem{20}S. Sartania, Z. Cheng, M. Lenzner, G. Tempea, Ch.
Spielmann, F.~Krausz, and K. Ferencz, ,,Generation of 0.1-TW 5-fs
optical pulses at a~1-kHz repetition rate", {\it Opt. Lett.}, vol.
22, pp. 1562--1564, 1997.
\bibitem{21}M. Nisoli, S. De Silvestri, O. Svelto, R. Szipoecs, K.
Ferencz, Ch. Spielmann, S. Sartania, and F. Krausz, ,,Compression
of high-energy laser pulses below 5 fs", {\it Opt. Lett.}, vol.
22, pp. 522--524, 1997.
\bibitem{22}S. A. Akhmanov, V. A. Vysloukh, and A. S. Chirkin, {\it Optics of femtose\-cond laser
pulses}. New York: American Institute of Physics, 1992.
\bibitem{23}A. Andreoni, M. Bondani, and M. A. C. Potenza,
,,Ultra-broadband and chirp-free frequency doubling in
$\beta$-barium borate", {\it Opt. Commun.}, vol.~154, pp.
376--382, 1998.
\bibitem{24}H. Wang, K. S. Wong, D. Deng, Z. Xu, G. K. L. Wong, and
J.~Zhang, ,,Kilohertz femtosecond UV-pumped visible $\beta$-barium
borate and lithium triborate optical parametric generator and
amplifier", {\it Appl. Opt.}, vol.~36, pp. 1889--1893, 1997.
\end{thebibliography}


\end{document}
