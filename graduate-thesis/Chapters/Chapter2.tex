\chapter{Background Theory}

Computational finance is a branch of applied computer science that deals with problems of practical interest in finance. Some slightly different definitions are the study of data and algorithms currently used in finance and the mathematics of computer programs that realize financial models or systems. Using computational finance in order to allocate assets in a portfolio is not at all unheard of and was first documented  1952.\cite{Markowitz:1952aa} Markowitz first introduced the concept of portfolio selection as an exercise in mean-variance optimisation. This required more computer power than was available at the time, so he worked on useful algorithms for approximate solutions. He theorised that risk-averse investors could construct portfolios to optimise or maximise expected return based on a given level of market risk, emphasising that risk is an inherent part of higher reward. According to his theory, it's possible to construct an "efficient frontier" of optimal portfolios offering the maximum possible expected return for a given level of risk. 

\section{Time Series Analysis}

A time series is a series of data points which may be indexed, listed, or graphed, in a time order. Most commonly, a time series is a sequence taken at successive equally spaced points in time. Thus it is a sequence of discrete-time data. Examples of time series are heights of ocean tides, counts of sunspots, and the daily closing value of the Dow Jones Industrial Average. Brockwell et al provide a formal description of time series as having a set of observations xt, each one being recorded at a specific time t. A discrete-time time series (the type to which this book is primarily devoted) is one in which the set T0 of times at which observations are made is a discrete set, as is the case, for example, when observations are made at fixed time intervals. Continuous time series are obtained when observations are recorded continuously over some time interval, e.g., when T0 = [0, 1].\cite{Peter-J.-Brockwell:2016aa}

Significant "time series momentum" has also been documented in equity index, currency, commodity, and bond futures for each of the 58 liquid instruments they consider.\cite{Moskowitz:2011aa} They find persistence in returns for 1 to 12 months that partially reverses over longer horizons, consistent with sentiment theories of initial under-reaction and delayed over-reaction. A diversified portfolio of time series momentum strategies across all asset classes delivers substantial abnormal returns with little exposure to standard asset pricing factors and performs best during extreme markets. Examining the trading activities of speculators and hedgers, they find that speculators profit from time series momentum at the expense of hedgers.

\subsection{Auto Regressive Moving Average (ARMA)}

In the statistical analysis of time series, autoregressive–moving-average (ARMA) models provide a parsimonious description of a weakly stationary stochastic process in terms of two polynomials, one for the autoregression and the second for the moving average. The notation ARMA(p, q) refers to the model with p autoregressive terms and q moving-average terms. This model contains the AR(p) and MA(q) models,

\(x_t = c+\varepsilon_t+\sum^{q}_{i=1} \varphi_iX_{t-i}\theta_i\varepsilon_{t-i}\)

Forecasting interest rates is of great concern for financial researchers, economists, and players in the fixed income markets. A study was carried out to develop an appropriate model for forecasting the short-term interest rates, implicit yield on 91 day treasury bill, overnight MIBOR rate, and call money rate.\cite{Radha:2015aa} The short-term interest rates are forecasted using univariate models such as the Random Walk, ARIMA, ARMA-GARCH, and ARMA-EGARCH. The appropriate model for forecasting is determined considering a six-year period from 1999. The results show that interest rates time series have volatility clustering effect and hence GARCH based models are more appropriate to forecast than the other models. Radha et al found that for commercial paper rate ARIMA-EGARCH model is the most appropriate model, while for implicit yield 91 day Treasury bill, overnight MIBOR rate, and call money rate, the ARIMA-GARCH model is the most appropriate model for forecasting.

\subsection{Auto Regressive Integrated Moving Average (ARIMA)}

In time series analysis, an autoregressive integrated moving average (ARIMA) model is a generalization of an ARMA model. Given a time series of data $X_t$ where t is an integer index and the $X_t$ are real numbers, an ARMA(p, q) model is given by

\(x_t - \alpha_1 X_{t-1} - \dots \alpha_{p'} X_{t-p'} = \varepsilon_t + \theta_1 \epsilon_{t-1} + \dots + \theta_q \epsilon_{t-q}\)

The aforementioned model is fitted to time series data either to better understand the data or to predict future points in the series (forecasting). ARIMA models are applied in some cases where data show evidence of non-stationarity, where an initial differencing step (corresponding to the "integrated" part of the model) can be applied to reduce the non-stationarity. 

The existence of weak-form efficiency in the Russian stock market is examined for the period 1st September 1995 to 1st May 2001 using daily, weekly and monthly Russian Trading System index time series.\cite{Abrosimova:2002aa} Several different approaches are used to assess the predictability of the RTS index time series. Unit root, autocorrelation and variance ratio tests are conducted for the null hypothesis of a random walk model. The results support the null hypothesis for the monthly data only. Further analysis is performed for the daily and weekly data. Linear and non-linear modelling of the serial dependence is conducted using ARIMA and GARCH models estimated on the in-sample period 1st September 1995 to 1st January 2001. Forecasts based on the best fitting models are performed for the out-of-sample period 2nd January 2001 to 1st May 2001. Comparisons of the forecasts reveal that none of the models outperforms the others, and the most accurate forecasts are obtained for just the first out-of-sample observation. Whilst our research results provide some limited evidence of short-term market predictability on the RTS, there is insufficient evidence to suggest that it would lead to a profitable trading rule, once transaction costs and risk are taken into account.

The main intention of this paper is to investigate, with new daily data, whether prices in the two Chinese stock exchanges (Shanghai and Shenzhen) follow a random-walk process as required by market efficiency.\cite{Darrat:2001aa} We use two different approaches, the standard variance-ratio test of Lo and MacKinlay (1988) and a model-comparison test that compares the ex post forecasts from a NAIVE model with those obtained from several alternative models (ARIMA, GARCH and Artificial Neural Network-ANN). To evaluate ex post forecasts, we utilize several procedures including RMSE, MAE, Theil's U, and encompassing tests. In contrast to the variance-ratio test, results from the model-comparison approach are quite decisive in rejecting the random-walk hypothesis in both Chinese stock markets. Moreover, our results provide strong support for the ANN as a potentially useful device for predicting stock prices in emerging markets.

\section{Statistical Machine Learning}

A machine learning algorithm is an algorithm that is able to learn through examples from data. To understand what an algorithm is, Cormen et al informally describe algorithms as "any well-defined computational procedures which takes some value, or set of values, as input and produce some value, or set of values, as output. An algorithm is thus a sequence of computational steps that transform the input into the output."\cite{Cormen:2009aa} In simple terms, it is possible to say that an algorithm is a sequence of steps which allow to solve a certain task. Similarly to a normal algorithm, a machine learning algorithm as defined formally by Tom M. Mitchell, states that "A computer program is said to learn from experience E with respect to some class of tasks T and performance measure P, if its performance at tasks in T, as measured by P, improves with experience E."\cite{Mitchell:1997aa}

\subsection{Classification}

Classifiers are a class of machine learning algorithms which identify in which category a new observation belongs to, on the basis of a training set of data containing observations whose category membership is known. A model based on discriminant analysis was sought out to categorise a stock as manipulated or non-manipulated based on certain key variables that capture the characteristics of the stock.\cite{Murugesan:2012aa} The model in which Murugesan et al chose, helps them identify stocks witnessing activities that are indicative of potential manipulation irrespective of the type of manipulation, such as action-based, information-based, or trade-based. The model which they proposed, helps investigators to arrive at a shortlist of securities that are potentially manipulated and which could be subject to further detailed investigation to detect the type and nature of the manipulation, if any. In a market like India, where there are about 5000 plus securities listed on its major exchanges, it becomes extremely difficult to monitor all securities for potential market abuse. Academics who have earlier used discriminant analysis have used the Linear Classification Function without validating the assumption that governs the model. Through their research, they have tested the assumption on data from the Indian capital market and found that the data does not comply with the assumptions that govern the use of the linear classification function. This therefore resulted in them using the Quadratic Classification Function, which is the appropriate technique for instances where the data does not meet the sated assumptions, to categorise stocks into two categories, namely manipulated and non-manipulated.

\subsection{Support Vector Machines}

Support vector machines (SVM), are a class of machine learning algorithms that have become incredibly popular in the past few years. SVMs are very similar to classifiers in the sense that they also classify data by drawing a line, called a decision boundary, to separate them. However, SVMs go a step further by calculating a vector from the data point with the smallest margin to the decision boundary. This is called a support vector. There exists vast research articles which predict the stock market as well pricing of stock index financial instruments but most of the proposed models focus on the accurate forecasting of the levels of the underlying stock index. There is a lack of studies examining the predictability of the direction of stock index movement. Given the notion that a prediction with little forecast error does not necessarily translate into capital gain, the authors of this research attempt to predict the direction of the S\&P CNX NIFTY Market Index of the National Stock Exchange, one of the fastest growing financial exchanges in developing Asian countries.\cite{Kumar:2016aa} Random forest and Support Vector Machines (SVM) are very specific type of machine learning method, and are promising tools for the prediction of financial time series. The tested classification models, which predict direction, include linear discriminant analysis, logit, artificial neural network, random forest, and SVM. Empirical experimentation suggests that the SVM outperforms the other classification methods in terms of predicting the direction of the stock market movement and random forest method outperforms neural network, discriminant analysis and logit model used in their study.

\subsection{Regression}

Regression analysis is widely used for prediction and forecasting to understand which among the independent variables are related to the dependent variable while also exploring the forms of these relationships. In restricted circumstances, regression analysis can be used to infer causal relationships between the independent and dependent variables. Regression analysis helps one understand how the typical value of the dependent variable changes when any one of the independent variables is varied, while the other independent variables are held fixed. Kakushadze et al provide a systematic quantitative framework in what is intended to be a “pedagogical” fashion for discussing mean-reversion and optimisation.\cite{Kakushadze:2015aa} In their paper, they start off their research with pair trading and add complexity by following the sequence “mean-reversion via demeaning, regression, weighted regression, (constrained) optimization, factor models”. They discuss in further detail how to conduct mean-reversion based on this approach, including common pitfalls encountered in practical applications, such as the difference between maximising the Sharpe ratio and minimising an objective function when trading costs are included. Kakushadze et al also discuss explicit algorithms for optimization with linear costs, constraints and bounds, and also illustrate their discussion on an explicit intraday mean-reversion alpha.

\subsection{Decision Trees}

Decision tree learning uses a decision tree as a predictive model which maps observations about an item, represented in the branches, to conclusions about the item's target value represented in the leaves. Tree models where the target variable can take a finite set of values are called classification trees; in these tree structures, leaves represent class labels and branches represent conjunctions of features that lead to those class labels. Decision trees where the target variable can take continuous values, typically real numbers, are called regression trees. Creamer et al propose a multi-stock automated trading system which relies on a layered structure consisting of a machine learning algorithm, an online learning utility, and a risk management overlay.\cite{Creamer:2010aa} An alternating decision tree (ADT), which is implemented with Logitboost, was chosen as their underlying algorithm. One of the strengths of their approach is that the algorithm is able to select the best combination of rules derived from well-known technical analysis indicators and is also able to select the best parameters of the technical indicators. Additionally, their online learning layer combines the output of several ADTs and suggests a short or long position. Finally, the risk management layer in which they implemented, can validate the trading signal when it exceeds a specified non-zero threshold and limit the application of their trading strategy when it is not profitable. They tested the expert weighting algorithm with data of 100 randomly selected companies of the S\&P 500 index during the period 2003–2005. They found that their algorithm generates abnormal returns during the test period. Their experiments show that the boosting approach was able to improve the predictive capacity when indicators were combined and aggregated as a single predictor. Even more, the combination of indicators of different stocks demonstrated to be adequate in order to reduce the use of computational resources, and still maintain an adequate predictive capacity.

\section{Bayesian Statistics}

Bayesian Statistics, a form of probabilistic programming, describes probabilistic models and then performs inference in those models. Probabilistic reasoning is a foundational technology of machine learning and has been used by companies such as Google, Amazon, and Microsoft. Probabilistic reasoning has been used for predicting stock prices, recommending movies, diagnosing computers, detecting cyber intrusions, and image detection. Gelman et al defines bayesian inference as the process of fitting a probability model to a set of data and summarising the result by a probability distribution on the parameters of the model and on unobserved quantities such as predictions for new observations.\cite{Gelman:2014aa}

\subsection{Markov Chain Monte Carlo (MCMC)}

In statistics, (MCMC) methods are a class of algorithms for sampling from a probability distribution based on constructing a Markov chain that has the desired distribution of its equilibrium distribution. The state of the chain after a number of steps is then used as a sample of the desired distribution. The quality of the sample improves as a function of the number of steps.

A 1993 paper was prepared for the purpose of presenting the methodology and uses of the Markov Chain Monte Carlo simulation technique as applied in the evaluation of investment projects to analyse and assess risk.\cite{Savvides:1994aa} In the first part of his paper, he highlights the importance of risk analysis in investment appraisal. The author follows by presenting the various stages in the application of the risk analysis process and examines the interpretation of the results generated by a risk analysis application including investment decision criteria and various measures of risk based on the expected value concept. In the final part of his paper, he draws some conclusions regarding the usefulness and limitations of risk analysis in investment appraisal.

Stochastic volatility models are increasingly important in practical derivatives pricing applications, yet relatively little work has been undertaken in the development of practical Monte Carlo simulation methods for this class of models. This paper considers several new algorithms for time-discretization and Monte Carlo simulation of Heston-type stochastic volatility models.\cite{Andersen:2007aa} The algorithms are based on a careful analysis of the properties of affine stochastic volatility diffusions, and are straightforward and quick to implement and execute. Tests on realistic model parameterizations reveal that the computational efficiency and robustness of the simulation schemes proposed in the paper compare very favourably to existing methods.

Bond yields today are well below and stock market valuations are well above their historical average.\cite{Blanchett:2013aa} There are no historical periods in the United States where comparable low bond yields and high equity valuations have occurred simultaneously. Both current bond yields and stock values have been shown to predict near-term returns. Portfolio returns in the first decade of retirement have an outsize impact on retirement income strategies. Traditional Monte Carlo simulation approaches generally do not incorporate market valuations into their analysis. In order to simulate how retirees will fare in a low return environment for both stocks and bonds, Blanchett et al incorporate the predictive ability of current valuations to simulate its impact on retirement portfolios. Blanchett et al estimate bond returns through an autoregressive model that uses an initial bond yield value where yields drift in the future. Blanchett et al use the cyclically adjusted price-to-earnings (CAPE) ratio as an estimate of market valuation to predict short-run stock performance. Our simulations indicate that the safety of a given withdrawal strategy is significantly affected by the initial bond yield and CAPE value at retirement, and that the relative impact varies based on the portfolio equity allocation. Using valuation measures current as of April 15, 2013, which is a bond yield of 2.0\% and a CAPE of 22, Blanchett et al find the probability of success for a 40\% equity allocation with a 4\% initial withdrawal rate over a 30 year period is approximately 48\%. This success rate is materially lower than past studies and has sobering implications on the likelihood of success for retirees today, as well as how much those near retirement may need to save to ensure a successful retirement.

This paper provides a numerical approach based on a Monte Carlo simulation for valuing dynamic capital budgeting problems with many embedded real options dependent on numerous state variables.\cite{Gamba:2003aa} Schwartz et al propose a way of decomposing a complex capital budgeting problem with many options into a set of simple options, suitably accounting for interaction and interdependence among them. The decomposition approach is numerically implemented using an extension of the Least Squares Monte Carlo algorithm, presented by Longstaff and Schwartz (2001) applied to our multi-option setting. Schwartz et al also provide a number of applications of our approach to well-known real options models and real life capital budgeting problems. Moreover, Schwartz et al present a set of numerical experiments to provide evidence for the accuracy of the proposed methodology.

Hoffman et al introduce the 'No-U-Turn-Sampler', an extension to the Hamiltonian Monte Carlo (HMC), which is an MCMC algorithm that avoids the random walk behavioir and sensitivity to correlated paramaters that plague many MCMC methods by taking a series of steps informed by first-order gradient information.\cite{Matthew-D.-Hoffman:2014aa} In their paper, they claim HMC’s performance to be highly sensitive to two user-specified parameters: a step size, and a desired number of steps. NUTS is an improvement from HMC as it eliminates the need to set a number of steps. Their algorithm uses a recursive algorithm to build a set of likely candidate points that spans a wide swath of the target distribution, stopping automatically when it starts to double back and retrace its steps.