\chapter{Experminetal Setup}

\section{Data Tidying}
A data set containing end of day stock prices, dividends, and splits for 3,000 US companies, curated by the Quandl community, and released into the public domain, was used. The date column in the CSV file was loaded into memory, and said column was converted to a date data type. The DataFrame was then sorted using the date column, starting from the oldest date, ending with the latest. The date column was also set to the index. The DataFrame was split into two, the training data set consisting of 80\%, and the test data set consisting of 20\% of the original DataFrame.

\section{Stock Selection}
The pairwise correlation of all the columns in the DataFrame was computing and stored in a new DataFrame. A list of stock pairs with low correlation were extracted.

\section{Time Series Analysis}
The selected stocks was extracted from the data set and stored in a DataFrame. The log returns of the stocks were calculated by calculating the logarithm of the stock's adjusted close price divided by the following day's adjusred close price. The resulting values from the calculation were then stored in a new column in the DataFrame and all infinite values were dropped from the series.

\subsection{Random Walk}
The first difference of the stocks were calculated abd stored in a new column in the Dataframe and all infinite valued were dropped.

\subsection{Ordinary Least Squares (OLS)}
The series was fitted to an OLS model. The 15 day SMA was used as a nobs x k array where nobs is the number of observations and k is the number of regressors, to train the model, while the adjusted close price of the stock was used as the dependent variable for the model to predict. The mean absolute error, mean squared error, median absolute error, and r2 score were used as metrics in order to rank the performance of the model's prediction capabilities in in-sample testing.

\subsection{Auto Regressive (AR)}
The series was fitted to an AR(p) model with a maximum lag value of 30. The IC was used to fit the AR model in order to select the optimal lag length. No constants were passed when fitting the AR model to the series. The optimal lag for the fit of the AR model was then calculated using the same paramaeters passed when fitting the AR model.

\subsection{Moving Average (MA)}
The series was fitted to an MA(p, q) model with an order selected based on the lowest AIC. A maximum lag of 30 was once again passed to the MA model with no constant. The exact loglikelihood for the fit of the MA model was maximized via the Kalman Filter.

\subsection{Auto Regressive Moving Average (ARMA)}
The series was fitted to an ARMA(p, q, r) model with an order selected based on the lowest AIC. No constants were passed to the ARMA model. The exact loglikelihood for the fit of the ARMA model was maximized via the Kalman Filter.

\subsection{Auto Regressive Integrated Moving Average (ARIMA)}
The series was fitted to an ARIMA(p, q, r) model with an order selected based on the lowest AIC. No constants were passed to the ARIMA model. The exact loglikelihood for the fit of the ARIMA model was maximized via the Kalman Filter.

\section{Machine Learning}
The data set was split for training and testing purposes when fitting and predicting data. The training data set consisted of 80\% of the whole data set, while the test data set consisted of 20\%.

\subsection{Classification}
2, 3, 4, 5, and 6 day SMAs were used as the features to train the models, while a binary value used to determine whether the current day's adjusted close has risen or not from the previous day was used as the target valued for the model to predict. In-sample testing was carried out using the models predict function, passing the test data set's features in order to predict the output. The classification report and confusion matrix were used as metrics in order to rank the performance of the model's prediction capabilities in in-sample testing. An algorithm was developed for out-of-sample testing. The algorithm iterates for a number of n steps, increasing the index by 1 each step, forecasting the next day's outcome, and calculating the SMAs. 

\subsubsection{Decision Tree}
The decision tree classifier was fit using the Gini impurity criterion to measure the quality of the split. The model was given the liberty to select the best strategy in order to split the tree at each node. The maximum allowed depth of the tree was left unrestricted, which was the same as for the maximum leaf nodes. A threshold of 1e-7 was used to terminate the tree growth to determine if a node is a leaf, if the impurity of a node is below the threshold, the node is a leaf. The minimum number of samples required to be at a leaf node was set to 1, while the minimum number of samples required to split an internal node was set to 2. The data fitted to the model was not pre-sorted, and the random number generator used by the model was that of Numpy's RandomState.

\subsubsection{Boosted Decision Tree}
The boosted decision tree was fit using the 'SAME.R' real algorithm which converges at a faster rate, achieving a lower test error with fewer boosting iterations. The maximum number of estimators at which boosting is terminated was set to 50; in case of perfect fit, the learning procedure is stopped early. The learning rate which is the contribution of each classifier of the model was shrunk by 1. The best estimator used to fit the data to the model was a Decision Tree Classifier, and the random number generator used by the model was that of Numpy's RandomState. 

\subsubsection{Support Vector Machine (SVM)}
The C-Support Vector Classification implementation was used for the SVM model with a 0 penalty parameter of the error term. A kernal type of 'rbf' was used when fitted the model to the data, along with a polynomial kernel function degree of 3. The gama 'rbf' Kernel coefficient was calculated by dividing the number of features by 1, while no probability estimates were used when fitting. A shrinking heuristic was used and a tolerance of 1e-3 was used for stopping criterion. A cache size of 200MB was used for the kernel when fitting the model, and all classes were assigned a weight of one. Verbose output was not used, and the random number generator used by the model was that of Numpy's RandomState. No limits were set on the iterations within the solver, and a  one-vs-rest decision function of shape (number of samples, number of classes) was returned.

\subsubsection{Random Forest}
The random forest was fit with bootstrap samples when building the trees, while all the weights associated were set to 1. The 'gini' function to measure the quality of a split, and no maximum depth of the tree was set, allowing the nodes to expand until all leaves are pure or until all leaves contain less than the minimum split samples. The number of features to consider when looking for the best split was the square root of the number of passed, and no limit on the maximum leaf nodes for growing trees was set. A threshold of 1e-7 was used to terminate the tree growth to determine if a node is a leaf, if the impurity of a node is below the threshold, the node is a leaf. The minimum number of samples required to be at a leaf node was set to 1, and the minimum number of samples required to split an internal node was set to 2. The minimum weighted fraction of the sum total of weights (of all the input samples) required to be at a leaf node was set to 0, and the number of trees in the forest was set to 10. The number of jobs to use for the computation was set to 1, making use of only 1 CPU core, and out-of-bag samples to estimate the generalization accuracy were not used. The verbosity of the tree building process was not controlled, and the random number generator used by the model was that of Numpy's RandomState. The model was built using a cold start by not making use of the previous call to fit and add more estimators to the ensemble, meaning a whole new forest was fit instead.

\subsubsection{K-Nearest Neighbour}
The k-nearest neighbour was fit with a number of 5 neighbors to use for k-neighbours queries, with a uniform weight making all points in each neighbourhood weighted equally when carrying out predictions. The model was left at liberty to select the most appropriate algorithm based on the values passed when fitting the model to the data. The lead size of the model was set to 30, and a minkowski distance metric with p equal to 2 which equivalent to the standard Euclidean metric. The number of jobs to use for the computation was set to 1, making use of only 1 CPU core.

\subsubsection{Logistic Regression}
The logistic regression was fit with an inverse of regularization of strength 0 specifying stronger regularisation, while all the weights associated were set to 1. Dual formulation was used in conjunction with the l2 penalty with a liblinear solver. A constant (bias or intercept) was added to the decision function, and the intercept scalar was set to 1. The maximum number of iterations taken for the solvers to converge were set to 100, while the multiclass option used was ‘ovr’, thus fitting binary problem for each label.  The number of jobs to use for the computation was set to 1, making use of only 1 CPU core, and the random number generator used by the model was that of Numpy's RandomState. The tolerance for stopping criteria was set to 0.0001, and verbosity was used with a value of 0 for the liblinear solver. The model was built using a cold start by not making use of the previous call to fit and add more estimators to the ensemble, meaning a whole new forest was fit instead.

\subsubsection{Bernoulli Naive Bayes}
The model was fit with an additive (Laplace/Lidstone) smoothing parameter of 0 for no smoothing, and the threshold for binarising (mapping to booleans) of sample features was set to 0, and was presumed to already consist of binary vectors. The model was set to learn class prior probabilities, which means that the priors were adjusted according to the data.

\subsubsection{Gaussian Naive Bayes}
The model was fit to the data with no previous prior probabilities of the classes, thus the priors were not adjusted according to the data.

\subsubsection{Neural Network}
The model was fit to the data with 3 hidden layer of 100 neurons each using the rectified linear unit activation function for the hidden layer. The L2 penalty parameter used, which is the regularisation term, was set to 0.0001, and the size of minibatches for stochastic optimisers was set to the minimum of the two arguments being the number 200 and the number of samples. The solver used for weight optimisation was the 'adam' solver, which refers to a stochastic gradient-based optimiser. Since the 'adam' solver was used, the exponential decay rate for estimates of first moment vector was set to 0.9, while the second moment vector was set to 0.999. Early stopping was not used to terminate training when validation score is not improving whilst fitting the data to the model, and the value for numerical stability in the 'adam' solver was set to 1e-08. A constant learning rate was used for weight updates in conjunction with the initial learning rate, which controls the step-size in updating the weights, and was set to 0.001. The maximum number of iterations which the solver iterates until convergence was set to 200, and the momentum for gradient descent update was set to 0.9. Nesterov's momentum was used, and the exponent for the inverse scaling learning rate, which is used in updating the effective learning rate, was set to 0.5. A random number generator was not used, and the samples were shuffled in each iteration. The tolerance for optimisation was set to 0.0001 when the loss or score is not improving by at least the tolerance score set for two consecutive iterations in which convergence is considered to be reached and training is stopped. The validation fraction, which is the portion of training data to set aside as validation set for early stopping, was set to 0.1. The model was built using a cold start by not making use of the previous call to fit as initialisation, meaning the previous solution was erased.

\subsubsection{Stochastic Gradient Descent}
The model was fit to the data using the ordinary least squares fit ‘squared epsilon insensitive’ loss function, which ignores errors less than epsilon and is linear past that; this is the loss function used in SVR. The penalty used, also referred to regularisation term,  was 'l2' which is the standard regularizer for linear SVM models, and the constant used to multiply the regularisation term was 0.0001. The elastic net mixing paramater was set to 0.15, while the fit intercept was estimated, meaning the data was not assumed to be already centered. A total of 5 passes over the training data, also known as epochs, were used, and the training data was shuffled after each epoch. No seed from the pseudo random number generated was used while shuffling the data, and a level 0 verbosity was used. The epsilon threshold in the epsilon-insensitive loss functions was set to 0.1, and any differences between the current prediction and the correct label were ignored if they were less than the threshold. The learning rate schedule used was 'invscaling', and the initial learning rate was set to 0.01. The exponent for inverse scaling learning rate was set to 0.25, and the model was built using a cold start by not making use of the previous call to fit as initialisation. The SGD weights of the model were not averaged.

\subsection{Regression}
15 and 50 day SMAs were used as the features to train the models, while the adjusted close price of the stock was used as the target valued for the model to predict. In-sample testing was carried out using the models predict function, passing the test data set's features in order to predict the output. The mean absolute error, mean squared error, median absolute error, and $R^{2}$ (coefficient of determination) score were used as metrics in order to rank the performance of the model's prediction capabilities in in-sample testing. An algorithm was developed for out-of-sample testing. The algorithm iterates for a number of n steps, increasing the index by 1 each step, forecasting the next day's outcome, and calculating the SMAs. 

\subsubsection{Decision Tree}
The decision tree model was fit with a mean squared error, which is equal to variance reduction as feature selection criterion, and “mae” for the mean absolute error. The maximum allowed depth of the tree was left unrestricted, which was the same as for the maximum leaf nodes. A threshold of 1e-7 was used to terminate the tree growth to determine if a node is a leaf, if the impurity of a node is below the threshold, the node is a leaf. The minimum number of samples required to be at a leaf node was set to 1, while the minimum number of samples required to split an internal node was set to 2. The data fitted to the model was not pre-sorted, and the random number generator used by the model was that of Numpy's RandomState. The model was given the liberty to select the best strategy in order to split the tree at each node.

\subsubsection{Boosted Decision Tree}
The boosted decision tree was fit with with a decision tree regressor as the base estimator from which the boosted ensemble is built. The maximum number of estimators at which the boosting is terminated was set to 50, and the learning rate which shrinks the contribution of each regressor was set to 1.0. The loss function used when updating the weights after each boosting iteration was set to linear, and the random number generator used by the model was that of Numpy's RandomState.

\subsubsection{Random Forest}
The random forest was fit with a mean absolute error, and bootstrap samples were used when building trees. The maximum features to consider when looking for the best split were left to the model to select the best number, and the number of jobs to run in parallel for both the fitting of the model and its predictions. The model was built using a cold start by not making use of the previous call to fit and add more estimators to the ensemble, meaning a whole new forest was fit instead. A threshold of 1e-7 was used to terminate the tree growth to determine if a node is a leaf, if the impurity of a ode is below the threshold, the node is a leaf. The minimum number of samples required to be at a leaf node was set to 1, while the minimum number of samples required to split an internal node was set to 2.

\subsubsection{Linear Regression}
The linear regression was fit with no normalised regressors, note that this makes the hyperparameters learnt more robust and almost independent of the number of samples. The number of jobs to use for the computation was set to 1, making use of only 1 CPU core. The intercept of the model was calculated which centres the data being fit to the model.

\subsubsection{Neural Network}
The model was fit to the data with the same paramaters as the Neural Network classifier.

\subsubsection{Stochastic Gradient Descent}
The model was fit to the data with the same paramaters as the SGD classifier.

\section{Bayesian Statistics}
The last 500 rows of the selected stocks were extracted from the data set and stored into a DataFrame. The log returns were calculated by dividing each day's adjusted close with the adjusted close of the following day, in logarithmic form. The resulting values from the said calculation were then stored in a new column in the DataFrame and all infinite values were dropped from the series. The sharpe ratio of the original and predicted price returns was calculated to serve as an accuracy score in the in-sample tests.

\subsection{No-U-Turn Sampler (NUTS)}
The model returns were modeled with a Student-t distribution with an unknown degrees of freedom paramater, and a scale paramater determined by a latent process. 

\subsection{Metropolis-Hastings}
This model was fit to the data with the same paramaters as the NUTS algorithm except that the model was optimised using the L-BFGS-B algorithm.

\section{Strategy}
The automated algorithmic strategies were run over the period of 01/05/2014. An ordered dictionary was used to store a series of data frames container the price data of the stocks selected. The data was tidied where the columns 'open', 'high', 'low', 'close', 'ex-dividend', and 'split\_ratio' were dropped from each data frame and the columns 'ticker', 'adj\_open', 'adj\_high', 'adj\_low', and 'adj\_close' were renamed to 'sid', 'open', 'high', 'low', and 'close' respectively. The ordered dictionary was converted into a panel and passed to the trading algorithm. An ordered dictionary was used to store a boolean value for each ticker to determine whether the stock has already been invested in or not. All boolean values were set to false. 

\subsection{Classification}
An ordered dictionary was used to store the day's open and close price for each stock as the backtester simulated each trading day. The algorithm was only allowed to run once there was enough price data, the amount chosen was 6. The 2, 3, 4, 5, and 6 day SMAs were calculated and each stored in an array. The six arrays were converted into an array of tuples, where the i-th tuple contains the i-th element from each of the argument sequences or iterables. An array was created to store a 1 if the cloase price increased from the open price, while a 0 if it decreased. The independent variables, the SMAs, and the dependant variables, the binary outcome for the particular stock, were passed to the machine learning algorithm to predict the next day's close price. An order of 10\% of the current portfolio value was placed on a stock if the algorithm predicted an increase in the following day's price. The boolean value to determine whether the stock has already been invested in was set to true. The stock was shorted 10\% of the current portfolio value if the algorithm predicted a decrease in the following day's price. The boolean value to determine whether the stock has already been invested in was set to false.

\subsection{Regression}
An ordered dictionary was used to store the day's close price for each stock as the backtester simulated each trading day. The algorithm was only allowed to run once there was enough price data, the amount chosen was 50. The 15 and 50 day SMAs were calculated and each stored in an array. The two arrays were converted into an array of tuples, where the i-th tuple contains the i-th element from each of the argument sequences or iterables. The independent variables, the SMAs, and the dependant variables, the close prices for the particular stock, were passed to the machine learning algorithm to predict the next day's close price. The next day's 15 and 50 day SMAs were calculated using the predicted price and the following day's price predicted again. This process was iterated over 100 times in order to predict the close price of the following 100 days. An order was placed on a stock if the difference of the last predicted price and the current day's predicted price was higher than 10. The amount of the stock was determined based on the difference, the higher the difference, the larger the percentage of the allocation. The boolean value to determine whether the stock has already been invested in was set to true. The stock was shorted if the difference of the last predicted price and the current day's predicted price was higher than -10. The amount of the stock was determined based on the difference, the higher the difference, the larger the percentage of the allocation. The boolean value to determine whether the stock has already been invested in was set to false.