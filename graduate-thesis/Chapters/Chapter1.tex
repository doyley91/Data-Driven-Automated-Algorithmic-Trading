\chapter{Introduction}

"Investing in stocks is just like gambling" - is a phrase commonly heard coming from people describing stock investing. But is it really just like gambling? Were all those investors who made billions from the stock market just lucky? To understand this, we must first review what it means to buy stocks. In the stock market, buying stocks means owning a share of the company. It entitles the stock holder to a claim on assets as well as a fraction of the profits which the company generates. It is unfortunately very common for investors to misunderstand this concept, often thinking of shares as simply a trading vehicle and forget that stock represents the ownership of a company. 

So how are stocks valued? The value of a company’s stock depends on a number of factors, and assessing the value is not an easy practice. Investors are constantly trying to assess the profit that will be left over to shareholders. This is why stock prices fluctuate. The outlook for business conditions is always changing, and so are the future earnings of a company. Since many people fail to understand this concept, it is far too common to believe that the short-term price movements of a company are random. However, it is the long-term price movements of a company which reflect the value of a company as it is supposed to be worth the present value of the profits it will make. A company can survive without profits in the short term, as long as expectations of future earnings exist. A company may try to fool investors in the beginning, however a company’s stock price will eventually be expected to show the true value of the firm.

So how is this all different to gambling? Gambling is a zero-sum game. It merely takes money from a loser and gives it to a winner. No value is ever created. By investing, we increase the overall wealth of an economy. As companies compete, they increase productivity and develop products that can make our lives better. This does not mean that stock investing cannot be a gamble, as it is extremely common for many people to skip the due diligence before spending a huge chunk of their life savings on stock, often losing it all in the process.

This thesis aims to disprove two hypotheses, the Efficient Market Hypotheses - EMH, and the Zero-Sum Game theory.

\section{Efficient Market Hypothesis - EMH}

In financial economics, the efficient market hypothesis (EMH) is an investment theory which states it is impossible for an investor to "beat the market" because stock market efficiency causes existing share prices to always incorporate and reflect all relevant information. In accordance to the EMH, stocks always trade at their fair value on stock exchanges and only react to new information or charges in discount rates, making it impossible for investors to make a profit by either purchasing undervalued stocks or selling stocks for inflated prices. This means that as such, it should be impossible to outperform the overall market through expert stock selection or market timing, and the only way an investor can possibly obtain higher returns is by purchasing riskier investments.

\subsection{Breaking down EMH}
Although it is a cornerstone of modern financial theory, the EMH is highly controversial and often disputed. Believers argue it is pointless to search for undervalued stocks or to try to predict trends in the market through either fundamental or technical analysis.

While academics point to a large body of evidence in support of EMH, an equal amount of dissension also exists. For example, investors such as Warren Buffett have consistently beaten the market over long periods of time, which  in itself is impossible by definition according to the EMH. Detractors of the EMH also point to events such as the 1987 stock market crash, when the Dow Jones Industrial Average (DJIA) fell by over 20\% in a single day, which was clear evidence that stock prices can seriously deviate from their fair values.

\subsection{What EMH means for investors}
Proponents of the EMH conclude that, because of the randomness of the market, investors would be better off by investing in a low-cost, passive portfolio such as one comprising of various low risk index funds. Data compiled by Morningstar Inc. through its June 2015 Active/Passive Barometer study supports the conclusion. Morningstar compared active managers’ returns in all categories against a composite made of related index funds and exchange-traded funds (ETFs). The study found that year-over-year, only two groups of active managers successfully outperformed passive funds more than 50\% of the time. These were U.S. small growth funds and diversified emerging markets funds.

In all of the other categories, including U.S. large blend, U.S. large value, and U.S. large growth, among others, investors would have fared better by investing in low-cost index funds or ETFs. While a percentage of active managers do outperform passive funds at some point, the challenge for investors is being able to identify which ones will do so. Less than 25\% of the top-performing active managers are able to consistently outperform their passive manager counterparts.

\section{Zero-Sum Game}

Zero-sum, not to be confused with Empty sum, or Zero game, is a mathematical representation of a situation found in game theory and economic theory, in which one person’s gain is equivalent to another’s loss, so the net change in wealth or benefit is zero. A zero-sum game may have as few as two players, or millions of participants.

Zero-sum games are found in game theory, but are less common than non-zero sum games. Poker and gambling are popular examples of zero-sum games since the sum of the amounts won by some players equals the combined losses of the others. Games such as chess and tennis, in which there is one winner and one loser, are also zero-sum games. In the financial markets, options and futures are examples of zero-sum games, excluding transaction costs. For every person who gains on a contract, there is a counter-party who loses.

\subsection{Zero-Sum Game \& Economics}

When the Zero-Sum Game theory is applied specifically to economics, there are multiple factors to consider in oder to understand a zero-sum game. A zero-sum game assumes a version of perfect competition and perfect information; that is, both opponents in the model have all the relevant information to make an informed decision. To take a step back, most transactions or trades are inherently non zero-sum games because when two parties agree to trade they do so with the understanding that the goods or services they are receiving are more valuable than the goods or services they are trading for it, after transaction costs. This is called positive-sum, and most transactions fall under this category.

Options and futures trading is the closest practical example to a zero-sum game scenario. Options and futures are essentially informed bets on what the future price of a certain commodity will be in a strict timeframe. While this is a very simplified explanation of options and futures, generally if the price of that commodity rises within that timeframe, you can sell the futures contract at a profit. Thus, if an investor makes money off of that bet, there will be a corresponding loss. This is why futures and options trading often comes with disclaimers to not be undertaken by inexperienced traders. However, futures and options provide liquidity for the corresponding markets and can be very successful, for the right investor or company that is.

It is important to note that the stock market overall is often considered a zero-sum game, which is a misconception, along with other popular misunderstandings. Historically and in contemporary culture the stock market is often equated with gambling, which is definitely a zero-sum game. When an investor buys a stock, it is a share of ownership of a company that entitles that investor to a fraction of the company's profits. The value of a stock can go up or down depending on the economy and a host of other factors, but ultimately, ownership of that stock will eventually result in a profit or a loss that is not based on chance or the guarantee of someone else's loss. In contrast, gambling means that somebody wins the money of another who loses it.

There are other such myths regarding the stock market, some of which include: falling stocks must go up again at some point and stocks that go up must come down, as well as that the stock market is exclusively for the extremely wealthy.