\chapter{Conclusion}

Three financial forecasting methods were presented in this report, two of which showed little to no potential of ever producing any statistically significant result when the correct methodology was applied. The third method, machine learning, showed some potential in the tests carried out, which is why this method was built into an automated algorithmic strategy to trade with. The algorithm proved to be successful in forecasting future prices, using both classification and regression methods. However, the backtesting proved this method to fail in forecasting price falls. Once this factor was removed from the equation, the algorithms were very successful and reported a profit by the end of the test. This is however not always ideal as stocks which could fall in price could be catastrophic to the strategy. A stop loss would be ideal in insuring that no positions are held in downward falling stocks. It was also evident that regression methods were more successful in forecasting future price movements when compared to classification methods. 

If there is anything that this report shows, is that profitable stock market prediction is an extremely tough problem. Even though the strategies reported a profit by the end of the backtest, they still did not beat the market. Whether it is at all possible to use such methods to outperform the market's returns, ultimately remains an open question. These findings support the Efficient Market Hypothesis, proving that casual investors are better off investing in passive buy and hold strategies consisting of index funds and ETFs. However, there was some evidence found showing that the Random Walk Hypothesis does not hold true for all cases, as some stocks did show signs of repeating trends.